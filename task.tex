

\documentclass{article}
\usepackage[utf8]{inputenc}
\usepackage[T2A]{fontenc}
\usepackage[russian]{babel}
\usepackage{graphicx}
\usepackage{amsfonts}
\usepackage{amsmath}
\usepackage{amssymb}

\title{ТФКП, M3238-39}
%\author{murzinanastasiia}
\date{\today}

% auto numeration as <section>.<task>
\renewcommand*{\theenumi}{\thesection.\arabic{enumi}}
\renewcommand{\labelenumi}{\theenumi}

% additional macroses
\providecommand{\abs}[1]{\left\lvert#1\right\rvert} 
\newcommand{\theIm}{\operatorname{Im}}
\newcommand{\conj}[1]{\overline{#1}}

\begin{document}

\maketitle

\section{Комлексные числа}
1.1 Решить уравнение $\bar{z} = z^{n-1}, (n \neq 2)$\\ \\
1.2 Доказать, что оба значения $\sqrt{z^2-1}$ лежат на прямой, проходящей через начало координат и параллельной биссектрисе внутреннего угла треугольника с вершинами в точках $-1, 1$ и $z$, проведённой из вершины $z$.\\ \\
1.3 Доказать, что $(^n\sqrt{z})^m$ ($n, m$ - целые числа, а $(n,m)$ - наибольший общий делитель) имеет $\frac{n}{(n, m)}$ различных значений \\ \\
1.4 Доказать $\vert 1 - \bar{z_1} z_2 \vert^2 - \vert z_1 - z_2 \vert ^2 = (1 - \vert z_1 \vert ^2) (1 - \vert z_2 \vert ^2)  $\\ \\
1.5 Доказать, что если $\vert z_1 + z_2 + z_3 \vert  = 0 $  и $\vert z_1 \vert = \vert z_2 \vert = \vert z_3 \vert= 1$, то точки $z_1, z_2, z_3$ являются вершинами правильного треугольника \\ \\
1.6 Изобразить область или прямую: \begin{itemize}
    \item $\vert z-2 \vert^2 - \vert z+2 \vert^2 > 3$;
    \item $log_{\frac{1}{2}}\frac{\vert z - 1 \vert + 4}{3\vert z - 1\vert -2} > 1$;
    \item ${Im}(\overline{z^2-z})=2-{Im} z$;
    \item $\vert z \vert - 3{Im} z=6$;
    \end{itemize}
    \\
1.7 Определить семейство линий в $z$-плоскости($-\infty < C < \infty$), заданных уравнениями:
\begin{itemize}
	\item $Re \dfrac{1}{z}=C$
	\item $Im \dfrac{1}{z}=C$
	\item $\dfrac{\vert z - z_1 \vert}{\vert z - z_2 \vert}=\lambda, (\lambda > 0)$
\end{itemize}
    
1.8 Доказать, что многочлен $f(x) = (\cos \alpha + x \sin \alpha)^n - \cos n\alpha - x\sin n\alpha$ делится на $x^2+1$.

1.9 Найти наибольшее и наименьшее расстояния от начала координат до линии $\vert z + \dfrac{1}{z}\vert = a, (a>0)$

1.10 Первоначальное значение $Arg f(z)$ при $z=2$ принято равным 0. Точка $z$ делает один оборот против часовой стрелки по окружности с центром в начале координат и возвращается в точку $z=2$. Считая, что  $Arg f(z)$ меняется непрерывно при движении точки $z$, указать значение $Arg f(2)$ после указанного поворота, если: \begin{itemize}
	\item $f(z) = \sqrt{z-1}$
	\item $f(z) = \sqrt{z^2+2z-3}$
	\item $f(z) = \sqrt{\dfrac{z-1}{z+1}}$
\end{itemize} \\

1.11 Доказать, что $\frac{x^{2m}-a^{2m}}{x^2-a^2}=\prod_{k=1}^{m-1}(x^2-2ax\cos \frac{k\pi}{m}+a^2)$. \\

1.12 Найти на сфере Римана образы окружностей с центром в начале координат \\

1.13 Найти на комплексной плоскости образ параллели с широтой $\phi, (\pi/2\leq\phi\leq\pi/2)$. \\

1.14 Найти суммы $S_n = 1 + \frac{\sin x}{\sin x}+\frac{\sin 2x}{\sin^2 x}+...+\frac{\sin nx}{\sin^nx},
\sigma_n = 1 + \frac{\cos x}{\sin x}+\frac{\os 2x}{\sin^2 x}+...+\frac{\cos nx}{\sin^nx}$

1.15 Решить систему уравнений \begin{cases} z^3+w^5=0 \\ z^2\bar{w}^4=1  \end{cases}

1.16 Найти все корни следующих уравнений:
\begin{eqnarray}
\sin z + \cos z = 2;\\
\sin z - \cos z = 3;\\
\sh z - \ch z = 2i;\\
2\ch z + \sh z = i;\\
\cos z = \ch z;\\
\cos z = i \sh 2z.
\end{eqnarray}

\section{Отображения}
2.1 Как действует отображение $e^z$ на прямую $x=y$, прямую $(y=const, x\in R)$, полосу $y\in(\phi, \psi), x\in R$ ?\\
2.2 Какая функция отображает полуплоскость $Im z > 0$ в окружность единичного радиуса с центром в начале координат, причём $z_0 \rightarrow (0; 0)$? \\
2.3 Найти образы координатных осей ОХ и ОУ при преобразовании $w= \dfrac{z+1}{z-1}$.\\
2.4 Найти линейное преобразование, оторбражающий треугольник с вершинами $0, 1, i$  на подобный ему с вершинами $0, 2, 1+i$.\\
2.5 Найти линейную функцию, отображающую круг $\vert z \vert < 1$  на круг $\vert w - w_0 \vert < R$ так, чтобы центры кругов соответсвовали друг другу и горизонтальный диаметр переходил в диаметр, образующий угол $\alpha$ с направлением дейтвительной оси.\\
2.6 Построить область на плоскости $w$, в которую отображается угол $0<\phi<\pi/4$ с помощью функции $w=\frac{z}{z-1}$.\\
2.7 Во что преобразуется окружность $\vert z \vert =1 $ при отображении $w=\frac{1-z}{z}$?




\end{document}
