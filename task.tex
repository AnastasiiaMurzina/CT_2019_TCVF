\documentclass{article}
\usepackage[utf8]{inputenc}
\usepackage[T2A]{fontenc}
\usepackage[russian]{babel}
\usepackage{graphicx}
\usepackage{amsfonts}
\usepackage{amsmath}
\usepackage{amssymb}

\title{ТФКП, M3238-39}
%\author{murzinanastasiia}
\date{\today}

% auto numeration as <section>.<task>
\renewcommand*{\theenumi}{\thesection.\arabic{enumi}}
\renewcommand{\labelenumi}{\theenumi}

% additional macroses
\providecommand{\abs}[1]{\left\lvert#1\right\rvert} 
\newcommand{\theIm}{\operatorname{Im}}
\newcommand{\conj}[1]{\overline{#1}}

\begin{document}

\maketitle

\section{Комлексные числа}

\begin{enumerate}
    \item Решить уравнение $\conj{z} = z^{n-1}$ ($n \neq 2$).
    \item Доказать, что оба значения $\sqrt{z^2-1}$ лежат на прямой, проходящей через начало координат и параллельной биссектрисе внутреннего угла треугольника с вершинами в точках $-1$, $1$ и $z$, проведённой из вершины $z$.
    \item Доказать, что $(\sqrt[n]{z})^m$ ($n$, $m$ "--- целые числа) имеет $\frac{n}{(n, m)}$ различных значений ($(n, m)$ "--- наибольший общий делитель).
    \item Доказать, что $\abs{1 - \conj{z_1} z_2}^2 - \abs{z_1 - z_2}^2 = (1 - \abs{z_1}^2)(1 - \abs{z_2}^2)$.
    \item Доказать, что если $\abs{z_1 + z_2 + z_3} = 0$ и $\abs{z_1} = \abs{z_2} = \abs{z_3} = 1$, то точки $z_1$, $z_2$, $z_3$ являются вершинами правильного треугольника.
    \item Изобразить область или прямую: 
        \begin{itemize}
            \everymath{\displaystyle} % makes math better
            \item $\abs{z-2}^2 - \abs{z+2}^2 > 3$;
            \item $\log_{\frac{1}{2}} \frac{\abs{z - 1} + 4}{3 \abs{z - 1} - 2} > 1$;
            \item $\theIm \conj{z^2-z} = 2 - \theIm z$;
            \item $\abs{z} - 3 \theIm z = 6$.
            \everymath{\textstyle}
        \end{itemize}
\end{enumerate}

\end{document}
