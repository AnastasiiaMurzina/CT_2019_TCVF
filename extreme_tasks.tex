

\documentclass{article}
\usepackage[utf8]{inputenc}
\usepackage[T2A]{fontenc}
\usepackage[russian]{babel}
\usepackage{graphicx}
\usepackage{amsfonts}
\usepackage{amsmath}
\usepackage{amssymb}

\title{ТФКП, M3238-39}
%\author{murzinanastasiia}
\date{\today}

% auto numeration as <section>.<task>
\renewcommand*{\theenumi}{\thesection.\arabic{enumi}}
\renewcommand{\labelenumi}{\theenumi}

% additional macroses
\providecommand{\abs}[1]{\left\lvert#1\right\rvert} 
\newcommand{\theIm}{\operatorname{Im}}
\newcommand{\conj}[1]{\overline{#1}}

\begin{document}
Каждое задание можно выполнять в группах. Каждый должен знать хоть какой-то разумный кусок математических выкладок.

Рассмотрим в некоторой области векторное поле $A=(\frac{dx}{dt}=A_x, \frac{dy}{dt}=A_y), A=A_x+iA_y$. Нас интересует семейство фазовых траекторий $y(x)$, которые необходимо визуализировать. Поток векторного поля через границу $\gamma$ области: $N = \int_\gamma (A, n)ds = \int_\gamma -A_ydx+A_xdy$ (второе равенство не определение - надо показать), введём дивергенцию для перехода к интегралу по площади: $div~A=\frac{\partial A_x}{\partial x} + \frac{\partial A_y}{\partial y}$ (применить формулу Грина). Аналогично для циркуляции $\Gamma=\int_\gamma (A, t) ds$ и ротора $rot~A=\frac{\partial A_y}{\partial x} - \frac{\partial A_x}{\partial y}$. Интеграл по сопряжению векторного поля есть мнимый вектор, состоящий из потока и циркуляции: $\Gamma+iN=\int_\gamma \overline{A(z)}dz$.

\section*{Обтекание тел}
$1\div5$ человек \\
Рассмотреть поле скоростей жидкости $V=(v_x, v_y)$ (как мы знаем, жидкости несжимаемы $div~V=0$) с соответствующим комплексным потенциалом $f(z)=u(x,y)+iv(x,y)$, где $V=\overline{f'(z)}$. Рассмотреть односвязную область в потоке жикости с заданной скоростью на бесконечности $V_\inf$, причём $f'(\inf) = \overline{V_\inf}$. Разложить в ряд на бесконечности ($c_{-1}$ будет интересовать особенно). Скорость на бесконечности считать заданной, условие на границе обтекаемой области не изменная комплексная составляющая скорости, потому что на границе скорость направлена по касательной. 

Теорема: Потенциал $w = f(z)$ обтекания тела конформно отображает область D на внешность отрезка, параллельного действительной оси.

Отсюда мы хотим рассмотреть конформное преобразование области на внешность отрезка $[0, 1]$. Проделать разложения комплексного потенциала для областей:
\begin{itemize}
\item круговой срез цилиндрического тела;
\item тела с эллиптическим срезом;
\item профиля Жуковского, для которого функция $w=\frac{1}{2}(z+\sqrt{z^2-a^2})$ конформно отображает профиль на внешность круга.
\end{itemize}

%\section*{Вращение}
%$1\div3$ человека \\


%\section*{Возмущение мембраны}
%$1\div4$ человека \\

\end{document}
